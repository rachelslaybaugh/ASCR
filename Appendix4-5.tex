\documentclass[12pt]{article}
\usepackage[top=1in, bottom=1in, left=1in, right=1in]{geometry}

\usepackage{setspace}
\onehalfspacing

\usepackage{amssymb}
\usepackage{amsthm}
\usepackage{epsfig}
\usepackage{times}
\renewcommand{\ttdefault}{cmtt}

\usepackage{amsmath}
\usepackage{graphicx}
\usepackage{tikz}
\usepackage{float}
\usepackage{xspace}
\usepackage{mathrsfs}
\usepackage[mathcal]{euscript}
\usepackage{color}
\usepackage{array}
\usepackage[pdftex,hidelinks]{hyperref}
\usepackage[parfill]{parskip}
\usepackage{multicol}
\usepackage{enumitem}
\usepackage{etoolbox}
\usepackage{subcaption}
\usepackage{cite}
\usepackage{titlesec}
\usepackage[capitalize,compress]{cleveref}

\patchcmd{\thebibliography}{\section*{\refname}}{}{}{}
\titleformat*{\section}{\Large\bfseries}
\titleformat*{\subsection}{\large\bfseries}
\titlespacing{\section}{0pt}{0pt plus 2pt minus 2pt}{0pt plus 2pt minus 2pt}
\titlespacing{\subsection}{0pt}{\parskip}{-\parskip}

% math syntax
\newcommand{\rvec}{\ensuremath{\vec{r}}}
\newcommand{\vecr}{\ensuremath{\vec{r}}}
\newcommand{\vecx}{\ensuremath{\vec{x}}}
\newcommand{\vecy}{\ensuremath{\vec{y}}}
\newcommand{\omvec}{\ensuremath{\hat{\Omega}}}
\newcommand{\vOmega}{\ensuremath{\hat{\Omega}}}

%%%%%%%%%%%%%%%%%%%%%%%%%%%%%%%%%%
%%%%%%%%%%%%%%%%%%%%%%%%%%%%%%%%%%
%%%%%%%%%%%%%%%%%%%%%%%%%%%%%%%%%%
%%%%%%%%%%%%%%%%%%%%%%%%%%%%%%%%%%
\begin{document}

%COVER PAGE

\pagenumbering{gobble}

\begin{center}
{\bf  NON-CLASSICAL
TRANSPORT THEORY WITH MULTISCALE PATH DISTRIBUTIONS IN REAL-WORLD APPLICATIONS}\\ \vspace{5pt}
\setlength{\unitlength}{1in}
\begin{picture}(6,.1)
\put(0,0) {\line(1,0){6.5}}
\end{picture}
\vspace{5pt}
Appendix 4: Facilities \& Other Resources
\end{center}



%----------------------------------------------------------------------
%----------------------------------------------------------------------
The proposed project will take advantage of the condo-model Savio computing cluster available at Berkeley (\href{http://research-it.berkeley.edu/services/high-performance-computing}{http://research-it.berkeley.edu/services/high-performance-computing}). The Berkeley personnel in this proposal have 24 nodes, each consisting of Dual Ivybridge E5-2670v2 2.5Ghz 10-core processors (480 total cores), 64GB 1866 Mhz Memory, FDR Infiniband HCA, on this machine. They also have the ability to access the full machine, which has 331 compute nodes. These computing resources will facilitate the fast simulations required for this project. These resources will be available throughout the duration of this project and will be adequate for the computing to be conducted in this project.

Computing resources and software licenses at other facilities...
\vspace{5pt}
\begin{center}
Appendix 5: Equipment
\end{center}

No equipment beyond the facilities listed above is required.

\end{document}
